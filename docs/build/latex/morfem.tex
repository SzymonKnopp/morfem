%% Generated by Sphinx.
\def\sphinxdocclass{report}
\documentclass[letterpaper,10pt,english]{sphinxmanual}
\ifdefined\pdfpxdimen
   \let\sphinxpxdimen\pdfpxdimen\else\newdimen\sphinxpxdimen
\fi \sphinxpxdimen=.75bp\relax
\ifdefined\pdfimageresolution
    \pdfimageresolution= \numexpr \dimexpr1in\relax/\sphinxpxdimen\relax
\fi
%% let collapsible pdf bookmarks panel have high depth per default
\PassOptionsToPackage{bookmarksdepth=5}{hyperref}


\PassOptionsToPackage{warn}{textcomp}
\usepackage[utf8]{inputenc}
\ifdefined\DeclareUnicodeCharacter
% support both utf8 and utf8x syntaxes
  \ifdefined\DeclareUnicodeCharacterAsOptional
    \def\sphinxDUC#1{\DeclareUnicodeCharacter{"#1}}
  \else
    \let\sphinxDUC\DeclareUnicodeCharacter
  \fi
  \sphinxDUC{00A0}{\nobreakspace}
  \sphinxDUC{2500}{\sphinxunichar{2500}}
  \sphinxDUC{2502}{\sphinxunichar{2502}}
  \sphinxDUC{2514}{\sphinxunichar{2514}}
  \sphinxDUC{251C}{\sphinxunichar{251C}}
  \sphinxDUC{2572}{\textbackslash}
\fi
\usepackage{cmap}
\usepackage[T1]{fontenc}
\usepackage{amsmath,amssymb,amstext}
\usepackage{babel}



\usepackage{tgtermes}
\usepackage{tgheros}
\renewcommand{\ttdefault}{txtt}



\usepackage[Bjarne]{fncychap}
\usepackage{sphinx}

\fvset{fontsize=auto}
\usepackage{geometry}


% Include hyperref last.
\usepackage{hyperref}
% Fix anchor placement for figures with captions.
\usepackage{hypcap}% it must be loaded after hyperref.
% Set up styles of URL: it should be placed after hyperref.
\urlstyle{same}


\usepackage{sphinxmessages}




\title{morfem}
\date{Dec 14, 2022}
\release{}
\author{Szymon Knopp}
\newcommand{\sphinxlogo}{\vbox{}}
\renewcommand{\releasename}{}
\makeindex
\begin{document}

\ifdefined\shorthandoff
  \ifnum\catcode`\=\string=\active\shorthandoff{=}\fi
  \ifnum\catcode`\"=\active\shorthandoff{"}\fi
\fi

\pagestyle{empty}
\sphinxmaketitle
\pagestyle{plain}
\sphinxtableofcontents
\pagestyle{normal}
\phantomsection\label{\detokenize{index::doc}}\phantomsection\label{\detokenize{index:morfem}}
\sphinxAtStartPar
\sphinxstylestrong{morfem}(\sphinxstyleemphasis{domain}, \sphinxstyleemphasis{a0}, \sphinxstyleemphasis{a1}, \sphinxstyleemphasis{a2}, \sphinxstyleemphasis{b}, \sphinxstyleemphasis{t\_a0=lambda t: 1},
\sphinxstyleemphasis{t\_a1=lambda t: t}, *t\_a2=lambda t: t**2\sphinxstyleemphasis{,}t\_b=lambda t: t*)

\begin{DUlineblock}{0em}
\item[] Solves finite element method problem defined as:
\item[] (\sphinxcode{\sphinxupquote{t\_a0}} * \sphinxcode{\sphinxupquote{a0}} + \sphinxcode{\sphinxupquote{t\_a1}} * \sphinxcode{\sphinxupquote{a1}} + \sphinxcode{\sphinxupquote{t\_a2}} * \sphinxcode{\sphinxupquote{a2}}) \sphinxstyleemphasis{x} = \sphinxcode{\sphinxupquote{t\_b}} * \sphinxcode{\sphinxupquote{b}}
\item[] using model order reduction algorithms.
\end{DUlineblock}
\begin{quote}\begin{description}
\sphinxlineitem{Parameters}\begin{description}
\sphinxlineitem{domain}{[}\sphinxstyleemphasis{vector, shape (I)}{]}
\sphinxAtStartPar
Ordered set of \sphinxstyleemphasis{I} domain points \sphinxcode{\sphinxupquote{t}}, that the problem should be solved for.

\sphinxlineitem{a0}{[}\sphinxstyleemphasis{sparse csc array, shape (N, N)}{]}
\sphinxAtStartPar
First part of a system matrix.

\sphinxlineitem{a1}{[}\sphinxstyleemphasis{sparse csc array, shape (N, N)}{]}
\sphinxAtStartPar
Second part of a system matrix.

\sphinxlineitem{a2}{[}\sphinxstyleemphasis{sparse csc array, shape (N, N)}{]}
\sphinxAtStartPar
Third part of a system matrix.

\sphinxlineitem{b}{[}\sphinxstyleemphasis{sparse csc array, shape (N, M)}{]}
\sphinxAtStartPar
Part of an impulse vector.

\sphinxlineitem{t\_a0}{[}\sphinxstyleemphasis{(float) \sphinxhyphen{}\textgreater{} float}{]}
\sphinxAtStartPar
Function returning coefficient for \sphinxcode{\sphinxupquote{a0}}. \sphinxcode{\sphinxupquote{t}} \sphinxhyphen{}\textgreater{} \sphinxcode{\sphinxupquote{1}} by default.

\sphinxlineitem{t\_a1}{[}\sphinxstyleemphasis{(float) \sphinxhyphen{}\textgreater{} float}{]}
\sphinxAtStartPar
Function returning coefficient for \sphinxcode{\sphinxupquote{a1}}. \sphinxcode{\sphinxupquote{t}} \sphinxhyphen{}\textgreater{} \sphinxcode{\sphinxupquote{t}} by default.

\sphinxlineitem{t\_a2}{[}\sphinxstyleemphasis{(float) \sphinxhyphen{}\textgreater{} float}{]}
\sphinxAtStartPar
Function returning coefficient for \sphinxcode{\sphinxupquote{a2}}. \sphinxcode{\sphinxupquote{t}} \sphinxhyphen{}\textgreater{} \sphinxcode{\sphinxupquote{t**2}} by default.

\sphinxlineitem{t\_b}{[}\sphinxstyleemphasis{(float) \sphinxhyphen{}\textgreater{} float}{]}
\sphinxAtStartPar
Function returning coefficient for \sphinxcode{\sphinxupquote{b}}. \sphinxcode{\sphinxupquote{t}} \sphinxhyphen{}\textgreater{} \sphinxcode{\sphinxupquote{t}} by default.

\end{description}

\sphinxlineitem{Returns}\begin{description}
\sphinxlineitem{(x, q, a0\_r, a1\_r, a2\_r, b\_r)}{[}\sphinxstyleemphasis{tuple of ndarrays}{]}\begin{itemize}
\item {} 
\sphinxAtStartPar
\sphinxstylestrong{x} \sphinxhyphen{} \sphinxstyleemphasis{shape (I, Nr, M)}, array of problem solutions, where \sphinxcode{\sphinxupquote{x{[}n{]}}} is a solution for the \sphinxstyleemphasis{n}\sphinxhyphen{}th point in the \sphinxcode{\sphinxupquote{domain}}

\item {} 
\sphinxAtStartPar
\sphinxstylestrong{q} \sphinxhyphen{} \sphinxstyleemphasis{shape (N, Nr)} projection base, product of model order reduction algorithm

\item {} 
\sphinxAtStartPar
\sphinxstylestrong{a0\_r} \sphinxhyphen{} \sphinxstyleemphasis{shape (Nr, Nr)}, reduced \sphinxcode{\sphinxupquote{a0}}, equal to \sphinxcode{\sphinxupquote{q.T @ a0 @ q}}

\item {} 
\sphinxAtStartPar
\sphinxstylestrong{a1\_r} \sphinxhyphen{} \sphinxstyleemphasis{shape (Nr, Nr)}, reduced \sphinxcode{\sphinxupquote{a1}}, equal to \sphinxcode{\sphinxupquote{q.T @ a1 @ q}}

\item {} 
\sphinxAtStartPar
\sphinxstylestrong{a2\_r} \sphinxhyphen{} \sphinxstyleemphasis{shape (Nr, Nr)}, reduced \sphinxcode{\sphinxupquote{a2}}, equal to \sphinxcode{\sphinxupquote{q.T @ a2 @ q}}

\item {} 
\sphinxAtStartPar
\sphinxstylestrong{b\_r} \sphinxhyphen{} \sphinxstyleemphasis{shape (Nr, M)}, reduced \sphinxcode{\sphinxupquote{b}}, equal to \sphinxcode{\sphinxupquote{q.T @ b}}

\end{itemize}

\end{description}

\sphinxlineitem{Example}
\sphinxAtStartPar
Given the finite element method problem defined as \sphinxstyleemphasis{(G \sphinxhyphen{} t2 C)X = tB}, it can be solved by calling:

\begin{sphinxVerbatim}[commandchars=\\\{\}]
\PYG{n}{x}\PYG{p}{,} \PYG{n}{q}\PYG{p}{,} \PYG{n}{g\PYGZus{}r}\PYG{p}{,} \PYG{n}{\PYGZus{}}\PYG{p}{,} \PYG{n}{c\PYGZus{}r}\PYG{p}{,} \PYG{n}{b\PYGZus{}r} \PYG{o}{=} \PYG{n}{morfem}\PYG{p}{(}\PYG{n}{domain}\PYG{p}{,} \PYG{n}{G}\PYG{p}{,} \PYG{n}{csc\PYGZus{}array}\PYG{p}{(}\PYG{n}{G}\PYG{o}{.}\PYG{n}{shape}\PYG{p}{)}\PYG{p}{,} \PYG{n}{C}\PYG{p}{,} \PYG{n}{B}\PYG{p}{,} \PYG{n}{t\PYGZus{}a2}\PYG{o}{=}\PYG{k}{lambda} \PYG{n}{t}\PYG{p}{:} \PYG{o}{\PYGZhy{}}\PYG{n}{t}\PYG{o}{*}\PYG{o}{*}\PYG{l+m+mi}{2}\PYG{p}{)}
\end{sphinxVerbatim}

\sphinxAtStartPar
Equivalent calls include, (among others):

\begin{sphinxVerbatim}[commandchars=\\\{\}]
\PYG{n}{morfem}\PYG{p}{(}\PYG{n}{domain}\PYG{p}{,} \PYG{n}{G}\PYG{p}{,} \PYG{n}{C}\PYG{p}{,} \PYG{n}{csc\PYGZus{}array}\PYG{p}{(}\PYG{n}{G}\PYG{o}{.}\PYG{n}{shape}\PYG{p}{)}\PYG{p}{,} \PYG{n}{B}\PYG{p}{,} \PYG{n}{t\PYGZus{}a1}\PYG{o}{=}\PYG{k}{lambda} \PYG{n}{t}\PYG{p}{:} \PYG{o}{\PYGZhy{}}\PYG{n}{t}\PYG{o}{*}\PYG{o}{*}\PYG{l+m+mi}{2}\PYG{p}{)}
\PYG{n}{morfem}\PYG{p}{(}\PYG{n}{domain}\PYG{p}{,} \PYG{n}{G}\PYG{p}{,} \PYG{n}{C}\PYG{p}{,} \PYG{n}{csc\PYGZus{}array}\PYG{p}{(}\PYG{n}{G}\PYG{o}{.}\PYG{n}{shape}\PYG{p}{)}\PYG{p}{,} \PYG{n}{B}\PYG{p}{,} \PYG{n}{t\PYGZus{}a0}\PYG{o}{=}\PYG{k}{lambda} \PYG{n}{t}\PYG{p}{:} \PYG{l+m+mi}{1}\PYG{p}{,} \PYG{n}{t\PYGZus{}a1}\PYG{o}{=}\PYG{k}{lambda} \PYG{n}{t}\PYG{p}{:} \PYG{o}{\PYGZhy{}}\PYG{n}{t}\PYG{o}{*}\PYG{o}{*}\PYG{l+m+mi}{2}\PYG{p}{,} \PYG{n}{t\PYGZus{}b}\PYG{o}{=}\PYG{k}{lambda} \PYG{n}{t}\PYG{p}{:} \PYG{n}{t}\PYG{p}{)}
\end{sphinxVerbatim}

\sphinxAtStartPar
If, for example \sphinxstyleemphasis{E} is needed for all the points in domain, where \sphinxstyleemphasis{E\_t = t(X\_transposed)B}, it can be calculated as:

\begin{sphinxVerbatim}[commandchars=\\\{\}]
\PYG{n}{e} \PYG{o}{=} \PYG{n}{np}\PYG{o}{.}\PYG{n}{zeros}\PYG{p}{(}\PYG{p}{(}\PYG{n}{domain}\PYG{o}{.}\PYG{n}{size}\PYG{p}{,} \PYG{n}{x}\PYG{o}{.}\PYG{n}{shape}\PYG{p}{[}\PYG{l+m+mi}{1}\PYG{p}{]}\PYG{p}{,} \PYG{n}{b\PYGZus{}r}\PYG{o}{.}\PYG{n}{shape}\PYG{p}{[}\PYG{l+m+mi}{1}\PYG{p}{]}\PYG{p}{)}\PYG{p}{)}
\PYG{k}{for} \PYG{n}{i} \PYG{o+ow}{in} \PYG{n+nb}{range}\PYG{p}{(}\PYG{n}{domain}\PYG{o}{.}\PYG{n}{size}\PYG{p}{)}\PYG{p}{:}
    \PYG{n}{e}\PYG{p}{[}\PYG{n}{i}\PYG{p}{]} \PYG{o}{=} \PYG{n}{domain}\PYG{p}{[}\PYG{n}{i}\PYG{p}{]} \PYG{o}{*} \PYG{n}{x}\PYG{p}{[}\PYG{n}{i}\PYG{p}{]}\PYG{o}{.}\PYG{n}{T} \PYG{o}{*} \PYG{n}{b\PYGZus{}r}
\end{sphinxVerbatim}

\end{description}\end{quote}



\renewcommand{\indexname}{Index}
\printindex
\end{document}